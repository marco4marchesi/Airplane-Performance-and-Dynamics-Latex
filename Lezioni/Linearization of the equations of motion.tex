\section{Linearization of the equations of motion}
Linearization comes in handy when one needs to capture the typical behavior of the aircraft. It is accurate only near to a reference condition, namely the reference condition around which the system is linearized, but allows to forecast the behavior through powerful methods like the eigen-analysis (which of course cannot be applied to the non-linear system).
\\
The linearization is applied to the generalized equilibrium. To linearize, a perturbative approach is used:
\begin{align}
    \underline{W}_P &= \underline{W}_{P_0} + \Delta \underline{W}_P & \underline{\dot W}_P &= \underline{\dot W}_{P_0} + \Delta \underline{\dot W}_P & \underline{r}_P &= \underline{r}_{P_0} + \Delta \underline{r}_P .
\end{align}
The reference conditions appear in such a way that the derivatives are removed from the system. Also, taking P coincident to the center of gravity G, the terms of the equation become:
\begin{equation}
    vari termini
\end{equation}

and the linearized equations result in
\begin{equation}
\begin{cases}
m(\Delta\underline{\dot v}_{G|I}+\underline{\omega}_{\beta|I_0}\times \Delta\underline{v}_{G|I}+\Delta\underline{\omega}_{\beta|I}\times\underline{v}_{G|I_0})=\Delta\underline{f}_a+\Delta\underline{f}_g\\
\underline{\underline{J}}_G\Delta\underline{\dot\omega}_{\beta|I}+\underline{\omega}_{\beta|I_0}\times \underline{\underline{J}}_G\Delta\underline{\omega}_{\beta|I}+\Delta\underline{\omega}_{\beta|I}\times\underline{\underline{J}}_G\underline{\omega}_{\beta|I_0} = \Delta \underline{m}_{G_a}
    \end{cases}
\end{equation}
\subsubsection{Forcing term analysis}
Decomposition of the forcing terms in reference plus perturbation values.
\begin{equation}
    \textit{Decomposition}
\end{equation}
Once the decomposition is written, the aim is to write the expressions of the perturbation aerodynamic coefficients using variables that are already appearing in the equations.
\\
The main question is: what kind of dependencies can be expected?
Basically all the variables that are already present in the formulation, namely velocity, angular velocity, control surfaces deflections and thrust throttle; and in addition we have also the properties of the fluid, namely the effects of compressibility (mach number) and viscosity (Reynolds number).
\\
The aerodynamic coefficients can be therefore written as follows:
\begin{align}
\label{eq: linear form of aerodynamic coefficient}
    \Delta C_i = &\left[\left.\frac{\partial C_i}{\partial \underline{v}_{P|I}^\beta}\right|_0\Delta\underline{v}_{P|I}^\beta+\left.\frac{\partial C_i}{\partial \underline{\omega}_{P|I}^\beta}\right|_0\Delta\underline{\omega}_{P|I}^\beta+\left.\frac{\partial C_i}{\partial \underline{\dot v}_{P|I}^\beta}\right|_0\Delta\underline{\dot v}_{P|I}^\beta\right]+
\end{align}
\begin{align*}
    & \left[\left.\frac{\partial C_i}{\partial \delta_e}\right|_0\Delta\delta_e+\left.\frac{\partial C_i}{\partial \delta_a}\right|_0\Delta\delta_a+\left.\frac{\partial C_i}{\partial \delta_r}\right|_0\Delta\delta_r+\left.\frac{\partial C_i}{\partial \delta_T}\right|_0\Delta\delta_T\right]+\\
    &\left[\left.\frac{\partial C_i}{\partial Ma}\right|_0\Delta Ma+\left.\frac{\partial C_i}{\partial Re}\right|_0\Delta Re\right]
\end{align*}
where the "i" subscript means generic coefficient.
\\
The components are deliberately separated into 3 groups because they are called differently:
\begin{itemize}
    \item The first group is called "stability derivatives"
    \item The second group is called "control derivatives"
    \item The third group is called "flight regime derivatives"
\end{itemize}
\subsection{Non-dimensional analysis}
Using a non-dimensionalization of the system that we got so far allows to extend the analysis to very different aircraft without considering the absolute size of the vehicle. This means that we can then compare performances of aircrafts with very different sizes. 
Let's define
\begin{itemize}
    \item $(u,\Delta v, \Delta w)$ as:
    \begin{equation}
        \begin{cases}
                   u = \frac{\Delta U}{U_0}\\ 
            \Delta v = \frac{\Delta V}{U_0}\\ 
            \Delta w = \frac{\Delta W}{U_0}\\ 
        \end{cases} 
    \end{equation}
    \item $(\Delta \hat{p},\Delta \hat{q},\Delta \hat{r})$
    \begin{equation}
        \begin{cases}
            \Delta \hat{p} = \frac{\Delta pb}{2U_0}\\
            \Delta \hat{q} = \frac{\Delta qc}{2U_0}\\
            \Delta \hat{r} = \frac{\Delta rb}{2U_0}
        \end{cases}
    \end{equation}
    \item $(u,\Delta v, \Delta w)$ as:
    \begin{equation}
        \begin{cases}
            \Delta \hat{\dot u} = \frac{\Delta \dot U}{U_0} c_1\\
            \Delta \hat{\dot v} = \frac{\Delta \dot V}{U_0} b_1\\
            \Delta \hat{\dot w} = \frac{\Delta \dot W}{U_0} c_1\\
        \end{cases} 
    \end{equation}
    where $b_1 = b/2U_0$ and $c_1 = c/2U_0$  
\end{itemize}
The quantity just defined are strictly non-dimensional by definition!
\\
Let's now introduce the hypothesis of longitudinal velocity much greater than the lateral velocity. This assumption let's the definition of sideslip angle and angle of attack as:
\begin{align}
    \Delta \beta & = \frac{\Delta V}{U_0} = \Delta v & \Delta \alpha & = \frac{\Delta W}{U_0} = \Delta w
\end{align}

Let's introduce also a new set  of scaled quantities, which this time are dimensional (not non-dimensional)

\begin{equation}
    \begin{cases}
               \dot u = \frac{\Delta \dot U}{U_0}\\
        \Delta \dot v = \frac{\Delta \dot V}{U_0}\\
        \Delta \dot w = \frac{\Delta \dot W}{U_0}\\
    \end{cases} 
\end{equation}

Now we are ready to start the non-dimensionalization process. We will obtain a set of equations that are strictly speaking non-dimensional, with the r.h.s. as unknown. We also want to express l.h.s. and r.h.s. with familiar quantities, which are typically available or interesting to measure. In particular the target quantities are
\begin{enumerate}
    \item scaled linear accelerations;
    \item non-dimensional velocity;
    \item dimensional angular rates;
    \item dimensional angular accelerations.
\end{enumerate}
\subsubsection{Vector form}
The quantities just defined can be expressed in vector form as follows:
\begin{itemize}
    \item  non-dimensional velocity
    \begin{equation}
        \Delta \underline{v}_{G|I}^\beta  = \begin{Bmatrix}
            \Delta U\\
            \Delta V\\
            \Delta W\\
        \end{Bmatrix} \Rightarrow 
        \Delta \underline{\hat{v}}_{G|I}^\beta  = \Delta \underline{v}_{G|I}^\beta \cdot \frac{1}{U_0} =\begin{Bmatrix}
            u\\
            \Delta v\\
            \Delta w\\
        \end{Bmatrix}=\begin{Bmatrix}
            u\\
            \Delta \beta\\
            \Delta \alpha\\
        \end{Bmatrix},
    \end{equation}
    
    \item non-dimensional angular rates
    \begin{equation}
        \Delta \underline{\omega}_{G|I}^\beta  = \begin{Bmatrix}
            \Delta p\\
            \Delta q\\
            \Delta r\\
        \end{Bmatrix} \Rightarrow 
        \Delta \underline{\hat{\omega}}_{G|I}^\beta  = \underline{\underline{D}}^\beta \cdot \frac{1}{2U_0}\Delta \underline{\omega}_{G|I}^\beta  =\begin{Bmatrix}
            \Delta p b_1\\
            \Delta q c_1\\
            \Delta r b_1\\
        \end{Bmatrix}=\begin{Bmatrix}
            \Delta \hat{p}\\
            \Delta \hat{q}\\
            \Delta \hat{r}\\
        \end{Bmatrix},
    \end{equation}

    \item
    \begin{equation}
        \Delta \underline{v}_{G|I}^\beta  = \begin{Bmatrix}
            \Delta \dot U\\
            \Delta \dot V\\
            \Delta \dot W\\
        \end{Bmatrix} \Rightarrow 
        \Delta \underline{\hat{\dot v}}_{G|I}^\beta  = \underline{\underline{E}}^\beta \cdot \frac{1}{2U_0^2}\Delta \underline{\dot v}_{G|I}^\beta  =\begin{Bmatrix}
            \hat{\dot u}\\
            \Delta \hat{\dot v}\\
            \Delta \hat{\dot w}\\
        \end{Bmatrix}=\begin{Bmatrix}
            \hat{\dot u}\\
            \Delta \hat{\dot \beta}\\
            \Delta \hat{\dot \alpha}\\
        \end{Bmatrix},
    \end{equation}
    
     \item scaled linear accelerations
     \begin{equation}
        \Delta \underline{\dot v}_{G|I}^\beta  = \begin{Bmatrix}
            \Delta \dot{U}\\
            \Delta \dot{V}\\
            \Delta \dot{W}\\
        \end{Bmatrix} \Rightarrow 
        \Delta \underline{\dot{\hat{v}}}_{G|I}^\beta  = \Delta \underline{\dot v}_{G|I}^\beta \cdot \frac{1}{U_0} =\begin{Bmatrix}
            \dot u\\
            \Delta \dot v\\
            \Delta \dot w\\
        \end{Bmatrix}=\begin{Bmatrix}
            \dot u\\
            \Delta \dot \beta\\
            \Delta \dot \alpha\\
        \end{Bmatrix},
    \end{equation}
\end{itemize}

where $\underline{\underline{D}}^\beta = \begin{bmatrix}
    b_1 & 0 & 0 \\ 0 & c_1 & 0 \\ 0& 0& b_1
\end{bmatrix}$ and $\underline{\underline{E}}^\beta = \begin{bmatrix}
    c & 0 & 0 \\ 0 & b & 0 \\ 0& 0& c
\end{bmatrix}$.
\subsection{Non-dimensionalization process}
Let's start from the momentum balance. Divide both sides by $\frac{1}{2}\rho_0u_0^2S$, which is a force dimensionally speaking.
For the moment of momentum balance, instead, we divide by $\frac{1}{2}\rho_0u_0^2 S\underline{\underline{D}}^\beta$.
\\
Defining 
\begin{equation}
    m_1  = \frac{m}{\frac{1}{2}\rho_0U_0S}
\end{equation}
(note that it's $U_0$, not $U_0^2$) it is possible to write the non-dimensional form of the momentum balance
\begin{equation}
    m_1(\underline{\dot{\hat{v}}}_{G|I}^\beta + \underline{\omega}_{\beta|I}^\beta \times \Delta \underline{\hat v}_{G|I}^\beta + \Delta \underline{\omega}_{\beta|I}^\beta \times \underline{\hat v}_{G|I}^\beta)=r.h.s.
\end{equation}
To perform the non-dimensionalization of the moment of momentum balance we need to introduce the following quantities:
\begin{align}
    J_{x_1} &= \frac{J_x}{\frac{1}{2}\rho_0U_0^2Sb} &
    J_{xz_1} &= \frac{J_{xz}}{\frac{1}{2}\rho_0U_0^2Sb} &
    J_{y_2} &= \frac{J_y}{\frac{1}{2}\rho_0U_0^2Sb} &
    J_{z_1} &= \frac{J_z}{\frac{1}{2}\rho_0U_0^2Sb} \\
    J_{x_2} &= \frac{J_x}{\frac{1}{2}\rho_0U_0^2Sc} &
    J_{xz_2} &= \frac{J_{xz}}{\frac{1}{2}\rho_0U_0^2Sc} &
    J_{y_1} &= \frac{J_y}{\frac{1}{2}\rho_0U_0^2Sc} &
    J_{z_2} &= \frac{J_z}{\frac{1}{2}\rho_0U_0^2Sc} 
\end{align}
and perform the actual non-dimensionalization \textbf{(which I will not write here now due to time restictions)}.

\subsection{Right hand side of the equations}
The right hand side of the momentum balance contains the aerodynamic forces, which non-dimensionalized become
\begin{equation}
    \Delta \underline{f}_a^\beta = \frac{\Delta \underline{f}_a^\beta}{\frac{1}{2}\rho_0U_0^2S}=2u \begin{Bmatrix}
        C_{x_0}\\
        C_{y_0}\\
        C_{z_0}
    \end{Bmatrix} +
    \begin{Bmatrix}
        \Delta C_x\\
        \Delta C_y\\
        \Delta C_z
    \end{Bmatrix}
\end{equation}
and the aerodynamic moment, which becomes:
\begin{equation}
    \underline{m}_{G_a}^\beta = \underline{\underline{D}}^{\beta^{-1}} \frac{\underline{m}_{G_a}^\beta}{\frac{1}{2}\rho_0U_0^2S} = 2u \begin{Bmatrix}
        C_{L_{G_0}}\\
        C_{M_{G_0}}\\
        C_{N_{G_0}}
    \end{Bmatrix}+
    \begin{Bmatrix}
       \Delta C_{L_{G}}\\
       \Delta C_{M_{G}}\\
       \Delta C_{N_{G}}
    \end{Bmatrix}
\end{equation}

The full expression, which I will not report here because it's too long, maybe I will in the future, is the expansion of the $\Delta C_i$ coefficient in the form that we saw in [\ref{eq: linear form of aerodynamic coefficient}], leading to a matrix form depending on the multiple variables appearing in that expression.
\\
\textbf{Here we introduce the hypothesis} of small sensitivity to Reynolds and Mach and therefore eliminate them from the formulation, but theoretically we should consider them too.

\subsection{The other equations - Kinematic Equilibrium}
As we said the equations that we just developed are not enough to resolve the system, as we need another 3 to get the 9 unknowns defined. Let's then focus on the three rotation equations
\begin{equation}
    \underline{\omega}_{G|I}^\beta = \underline{\underline{S}}_{321}^\beta \dot e_{321}^\beta
\end{equation}

linearizing it we obtain
\begin{equation}
    \underline{\omega}_{G|I_0}^\beta + \Delta\underline{\omega}_{G|I}^\beta = \underline{\underline{S}}_{321_0}^\beta \underline{\dot e}_{321_0}^\beta + \underline{\underline{S}}_{321_0}^\beta \Delta\underline{\dot e}_{321}^\beta + \Delta \underline{\underline{S}}_{321}^\beta \underline{\dot e}_{321_0}^\beta + \Delta\underline{\underline{S}}_{321}^\beta \Delta \underline{\dot e}_{321}^\beta
\end{equation}
A couple of remarks: the second order term is negligible and the first term of the r.h.s. is constant near the reference condition.
\\
It is useful to invert the system to express the Euler angles as functions of the angular rates, as we did in the dynamic equilibrium.
\\
In this way we get straight to the 9 equations system that we were seeking, were the unknowns are differential, meaning that \textbf{we are now searching for the perturbations of the values near the reference condition. Not the value of the states themselves.}
\subsection{Applications of the linearized system}
The linearized form of this system has the following applications:
\begin{itemize}
    \item Eigen-analysis, which allows to assess dynamic stability and performance of the controls.
    \item Control system design, which allow or model based techniques (e.g. LQR)
    \item Time marching simulations, which allow to simulate integrating in time including also the controls.
\end{itemize}

\subsection{Introduction of the most common geometrical hypotheses}
The most common assumptions for an aircraft are:
\begin{itemize}
    \item winged;
    \item vertical plane of symmetry;
    \item tail to the back;
    \item horizontal tail.
\end{itemize}
These allow to neglect most of the coefficient derivatives, since the effect of lateral components will not appear in longitudinal dynamics and viceversa.
\\
Furthermore, a lot of the coefficients depend only on the reference condition, which implies that we can choose the reference condition to simplify the expressions, namely:
\begin{itemize}
    \item $p_0 = r_0 = 0$, which means that the plane of symmetry does not change is orientation (no roll rate or yaw rate, only pitch rate allowed)
    \item $v_0 = \beta_0 = 0$, which means that the plane of symmetry is not displacing laterally (no sideslip)
    \item $\varphi_0 = 0$, the rolling attitude of the plane of symmetry is null.
\end{itemize}
Of course not all flights comply with these reference conditions, but many do. Steady and unsteady, horizontal and climb symmetric flight comply with these hypotheses, as well as pitch-up maneuvers.
\\
Also, these hypotheses are also compatible with the general (not linearized) formulation for linear equilibrium, but the set of hypotheses at the beginning is different.
\\
Remark that the fact that some quantities are null in the reference condition does not mean that the perturbation is also null, don't be fooled.
\\
The system can be further simplified by introducing other three hypotheses on the dynamic components:
\begin{itemize}
    \item $C_{Y_0}=0$, no lateral force in the reference condition;
    \item $C_{L_0}=0$, no rolling moment in the reference condition;
    \item $C_{N_0}=0$, no yawing moment in the reference condition.
\end{itemize}
Here is when things get interesting: with the hypotheses introduced we can finally \textbf{decouple} the system into longitudinal and lateral-directional dynamics.
\\
The longitudinal dynamics are 4 equations that describe the effects of $F_x, F_z, M_{G_y},\Delta \dot \theta$,  and depend only on $u, \Delta \alpha, \Delta q, \Delta \theta, \delta_e, \delta_T$; the lateral-directional dynamics 5 equations that describe the effects of $F_y, M_{G_x}, M_{G_z}, \Delta \dot \varphi, \Delta \dot \psi$ and depend only on $\Delta \beta, \Delta p, \Delta r, \Delta \varphi, \Delta \psi, \delta_a, \delta_r$.
\subsection{State space form}
We can rearrange the system into the classical state space form by defining the state
\begin{equation}
    \underline{x}_L = \begin{Bmatrix}
        u, \Delta \alpha, \Delta q, \Delta \dot \theta
    \end{Bmatrix}^T,
\end{equation}
the control action
\begin{equation}
    \underline{u}_L = \begin{Bmatrix}
        \delta_e, \delta_T
    \end{Bmatrix}^T,
\end{equation}
and the system matrices
\begin{equation}
    \underline{\underline{M}}_L = \begin{bmatrix}
        m_1 -c_1C_{x_{\hat{\dot u}}}    & -c_1 C_{x_{\hat{\dot \alpha}}}        & 0         & 0\\
        -C_{z_{\hat{\dot u}}}           & m_1-c_1 C_{z_{\hat{\dot \alpha}}}     & 0         & 0\\
        -c_1C_{m_{G_{\hat{\dot u}}}}    & -c_1 C_{m_G{_{\hat{\dot \alpha}}}}    & J_{y_1}   & 0\\
        0                               & 0                                     & 0         & 1
    \end{bmatrix}
\end{equation}
\begin{equation}
    \underline{\underline{K}}_L = \begin{bmatrix}
        complete    &complete       & complete         & complete \\
       complete           & complete     & complete         & complete \\
        complete    & complete    &complete  & 0 \\
        0                               & 0                                     & -1         & 0
    \end{bmatrix}
\end{equation}
\begin{equation}
    \underline{\underline{U}}_L = \begin{bmatrix}
        C_{x_{\delta_e}}    &C_{x_{\delta_T}}  \\
      C_{z_{\delta_e}}        & C_{z_{\delta_T}}  \\
       C_{m_{G_{\delta_e}}}   & C_{m_{G_{\delta_T}}}  \\
        0   & 0 \\
    \end{bmatrix}
\end{equation}
therefore writing the matrix lagrangian form:
\begin{equation}
    \underline{\underline{M}}_L \underline{\dot x}_L + \underline{\underline{K}}_L \underline{x}_L = \underline{\underline{U}}_L \underline{u}_L
\end{equation}
which can be rewritten in state space form as
\begin{equation}
    \underline{\dot x}_L = \underline{\underline{A}}_L \underline{x} + \underline{\underline{B}}_L \underline{u}_L
\end{equation}
where
\begin{align}
    \underline{\underline{A}}_L &= -\underline{\underline{M}}_L^{-1}\underline{\underline{K}}_L & \underline{\underline{B}}_L &= \underline{\underline{M}}_L^{-1}\underline{\underline{U}}_L 
\end{align}

The same procedure can be applied to the lateral-directional dynamics, leading to a 5-states state-space system.
\\
Remark that the state space form is the gateway to the eigen-analysis and to control design, therefore very useful and important.
Note that matrices depend on aerodynamic characteristics, inertia characteristics and components in the reference condition. \textbf{All of these quantities influence the eigen-analysis and the control design}. Another remark is that the eigen-analysis focuses on the state matrix, it does not depend on controls.
\subsection{Effects of yaw on the system}
The perturbation on the yaw angle $\Delta \psi$ is actually not included in the equations, its behavior is just a consequence of the other quantities and therefore can be integrated separately (in post processing) and remove its equation from the equations of the lateral-directional dynamics. It can be seen as decoupled from the other components.

